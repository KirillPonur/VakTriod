\input{text/diss}

\begin{document}

\def\labauthors{Войтович Д.А., Понур К.А.}
\def\labgroup{440}
\def\labnumber{3}
\def\labtheme{Исследование принципов статического управления электронным потоком в триоде}
\renewcommand{\vec}{\mathbf}
\renewcommand{\phi}{\varphi}
\renewcommand{\hat}{\widehat}

\input{text/titlepage}

\section{Теоретическая часть}
\subsection{Устройство и принцип работы триода}
Триод отличается от диода наличием дополнительного электрода - сетки, расположенной между катодом и анодом. Поэтому анодный ток триода $J_a$ может управляться не только изменением потенциала анода $U_a$, но и изменением потенциала сетки $U_c$, т.е

\begin{equation}
	J_a=J_a (U_c,U_a).
\end{equation}

\begin{figure}[h!]
	\centering
	\includegraphics[width=0.8\linewidth]{fig/img3.jpg}
	\caption{Конструкция триода}
	\label{fig:1}
\end{figure}
Триод является усилителем тока, а при определенных условиях, - и напряжения. Основным рабочим режимом триода является, как и в диоде, режим ограничения тока пространственным зарядом. 

\subsection{Вольт-амперные характеристики триода}

Для описания свойств триода необходимо знать зависимости сеточного и анодного токов от подаваемых напряжений:
\begin{equation}
	J_a=J_a (U_c,U_a).
\end{equation}

\begin{equation}
	J_c=J_c (U_c,U_a).
\end{equation}

Наиболее важной для практических приложений является зависимость для анодного
тока. Поскольку она является двумерной, для ее описания используют две
одномерные зависимости:
\begin{equation}
	J_a=J_a (U_a), U_c=const
\end{equation}
и
\begin{equation}
	J_a=J_a (U_c), U_a=const
\end{equation}

Первая зависимость называется анодной характеристикой, вторая анодно-сеточной.

Можно показать, что для поля электродов плоского триода:
\begin{equation}
	E_{\text{эл}} = \frac{U_c+DU_a}{d_{kc}[1+D(1+d_{ca}/d_{kc})]},
\end{equation}
где 
\begin{equation}
	D \approx \frac{s}{2\pi d_{ca}}\ln{\frac{s}{2\pi \rho}}
\end{equation}

Коэффициент D определяет ту часть электрического поля вблизи катода, которая
обусловлена присутствием анода и связана с <<провисанием>> потенциала $U_a$ через
промежутки между витками сетки. Чем теснее расположены витки и чем толще
проволока сетки, тем меньше D. Это позволяет назвать D проницаемостью сетки.
Обычно $D \ll 1$, поэтому можно приближенно положить
\begin{equation}
	E = \frac{1}{d_{kc}}(U_c+DU_a)
\end{equation}

Тогда катодный ток в плоском триоде:
\begin{equation}
J_k=\frac49 \varepsilon_0 \sqrt{2 \eta} S_a \frac{(U_c+DU_a)^{3/2}}{d_{kc}^2}, 
\end{equation}
или в общем случае
\begin{equation}
J_k=P(U_c+DU_a)^{3/2}, 
\end{equation}
где  P-превеанс триода.

\subsection{Параметры триода}
Триод принято характеризовать тремя параметрами. Величина
\begin{equation}
	S= \pdv{J_a}{U_c} \eval _{U_a}, 
\end{equation}
является отношением бесконечно малых приращений анодного тока и напряжения на сетке и называется крутизной триода. 

Параметр 
\begin{equation}
	R= \pdv{U_a}{J_a} \eval _{U_c}, 
\end{equation}
называется внутренним сопротивлением. Геометрически этот параметр характеризует наклон анодных характеристик триода,
т.е. кривых, выражающих зависимость анодного тока от анодного напряжения при постоянном потенциале  
сетки.

Важным свйством триода, определяющим его способность усиливать напряжение является то, что малые 
изменения потенциала сетки эквивалентны в смысле воздействия на анодный ток большим изменениям 
потенциала анода, т.к. анод экранирован сеткой от катода и расположен от последнего дальше, 
чем сетка. Это свойство характеризуется коэффициентом усиления по напряжению:
\begin{equation}
	\mu= \pdv{U_a}{U_c} \eval _{J_a}, 
\end{equation}

Статические параметры не являются независимыми, а связаны соотношением:
\begin{equation}
	\mu= S \cdot R_i, 
\end{equation}
которое называется внутренним уравнением триода.

\section{Экспериментальная часть}
\subsection{Экспериментальная установка}
Схема установки:
\begin{figure}[h!]
	\centering
	\includegraphics[width=0.5\linewidth]{fig/img4.jpg}
	\caption{Схема экспериментальной установки}
	\label{fig:10}
\end{figure}

Блок режимов (1) обеспечивает требуемые напряжения на электродах исследуемой электронной 
лампы-триода. Исследуемый триод размещается на верхней панели измерительного блока (2), 
там же расположены элементы коммутации, позволяющие изменять схему включения триода. Для 
контроля режима лампы по постоянному току используются стрелочные лабораторные вольтметры 
и миллиамперметр ЛМ-1 (3). Звуковой генератор Г3-112 (4) обеспечивает подачу переменного 
напряжения для реализации метода переменной составляющей. Для измерения значений переменного 
напряжения используется аналоговый милливольтметр переменного тока В3-38 (5).

\subsection{Задание 1}
\begin{figure}[h!]
 	\centering
	\includegraphics[width=0.5\linewidth]{fig/img5.jpg}
 	\caption{К заданию 1}
 	\label{fig:11}
\end{figure}

Проведите измерения следующих зависимостей:

а) Ia = f(Uс) при Uа = 220 В; 

б) Ia = f(Uс) при Uа = 150 В;

в) S = f(Uс) при Uа = 220 В; 

г) S = f(Uа) при Uс = -8 В.

\subsection{Задание 2}
\begin{figure}[h!]
 	\centering
	\includegraphics[width=0.5\linewidth]{fig/img6.jpg}
 	\caption{К заданию 2}
 	\label{fig:11}
\end{figure}

Проведите измерения следующих зависимостей:

а) Ia = f(Uа) при Uс = - 8 В;

б) Ia = f(Uа) при Uс = - 6 В;

в) Ri = f(Uа) при Uс = - 8 В;

г) Ri = f(Uc) при Ua = 220 В.

\subsection{Задание 3}
По результатам измерений постройте графики зависимости от напряжения на сетке и на 
аноде ламы:

\newpage
а) анодного тока
\begin{figure}[h!]
	\centering
    \includegraphics[width=0.7\linewidth]{scripts/fig1}
	\caption{}
	\label{fig:10}
\end{figure}

\begin{figure}[h!]
	\centering
    \includegraphics[width=0.7\linewidth]{scripts/fig4}
	\caption{}
	\label{fig:10}
\end{figure}

\newpage
б) крутизны
\begin{figure}[h!]
	\centering
    \includegraphics[width=0.7\linewidth]{scripts/fig2}
	\caption{}
	\label{fig:10}
\end{figure}

\begin{figure}[H]
	\centering
    \includegraphics[width=0.7\linewidth]{scripts/fig3}
	\caption{}
	\label{fig:10}
\end{figure}

\newpage
в) внутреннего сопротивления
\begin{figure}[h!]
	\centering
    \includegraphics[width=0.7\linewidth]{scripts/fig5}
	\caption{}
	\label{fig:10}
\end{figure}

\begin{figure}[H]
	\centering
    \includegraphics[width=0.7\linewidth]{scripts/fig6}
	\caption{}
	\label{fig:10}
\end{figure}

\newpage
г) статического коэффициента усиления по напряжению
\begin{figure}[h!]
	\centering
    \includegraphics[width=0.65\linewidth]{scripts/fig7}
	\caption{}
	\label{fig:7}
\end{figure}

График \ref{fig:7} хорошо согласуется с теорией, поскольку при достижении определенного значения напряжения на аноде, смещение на сетке становится несущественным и перестает искривлять потенциальный рельеф в пространстве между катодом и сеткой, после этого статический коэффициент усиления по напряжению возрастает почти линейно с Ua.

\begin{figure}[H]
	\centering
    \includegraphics[width=0.7\linewidth]{scripts/fig8}
	\caption{}
	\label{fig:8}
\end{figure}

График \ref{fig:8} выглядит таким образом, потому что мю при средних потенциалах сетки является примерно постоянной величиной, а при уменьшении, то есть при приближении к величине запирания, мю уменьшается, таким образом проявляется "островной" эффект.

Поскольку рис.\ref{fig:7},\ref{fig:8}  строится по результатам двух
экспериментов на двух разных схемах, то добиться одиноков значений
$U_{\text{с}}$ и $U_{\text{а}}$ довольно сложно. Поэтому прежде чем считать
$\mu$ графики для крутизны и внутреннего сопротивления линейно
интерполировались, чтобы в дальнейшем можно было работать с одинаковыми
точками.


Из рис. \ref{fig:7} при достижении определенного значения напряжения на аноде,
смещение на сетке становится несущественным и перестает искривлять
потенциальный рельеф в пространстве между катодом и сеткой, после этого
статический коэффициент усиления по напряжению возрастает почти линейно с
$U_a$.

Если рассматривать рис. \ref{fig:8},то видно, что при средних потенциалах сетки
является примерно постоянной величиной, а при уменьшении, то есть при
приближении к величине запирания, $\mu$ уменьшается, таким образом проявляется
"островной" эффект.

Погрешность анодного напряжения $\Delta U_a = 2 B$.

Погрешность анодного тока $\Delta J_a = 0.1 mA$.

Погрешность сеточного напряжения $\Delta U_c = 0.1 B$.

Формулы для рассчета погрешности крутизны:
\begin{gather*}
	S=\frac{U_{out}}{U_{in}} \frac{1}{R_a}, \\
	\Delta S=  \frac{1}{R_a U_{in}}(\Delta U_{out}+\frac{U_{out}}{U_{in}}\Delta U_{in}). 
\end{gather*}

и внутреннего сопротивления:
\begin{gather*}
	R=\frac{U_{in}}{U_{out}} R_a, \\
	\Delta R=  \frac{R_a}{U_{out}}(\Delta U_{in}+\frac{U_{in}}{U_{out}}\Delta U_{out}). 
\end{gather*}

\end{document}
